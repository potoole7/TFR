\documentclass{article}

\usepackage[top=3cm, bottom=3cm, left=3cm,right=3cm]{geometry}
\usepackage[colorinlistoftodos]{todonotes}
\usepackage{graphicx}
\usepackage{bm}
\usepackage{titlesec}
\usepackage{amssymb}
\usepackage{amsmath}
\usepackage{bbm}
\usepackage{todonotes}
\usepackage{pdflscape}
\usepackage{caption}
\usepackage{subcaption}
\usepackage[T1]{fontenc}
\usepackage[utf8]{inputenc}
\usepackage{authblk}
\usepackage{pdfpages}
\usepackage{setspace} 
\usepackage{booktabs}
\usepackage{longtable}
\usepackage{float}
\usepackage{tikz}
\usepackage[colorlinks=true,citecolor=blue, linkcolor=blue]{hyperref}
\usepackage{multirow}
\usepackage{todonotes}
\usepackage{natbib} % print author's name and year when citing
\setlength{\tabcolsep}{5pt}
%%\setlength{\parindent}{0pt}
\usepackage[parfill]{parskip}
\renewcommand{\arraystretch}{1.5}

% \renewcommand\Affilfont{\itshape\footnotesize}
% \def\ci{\perp\!\!\!\perp}

% \renewcommand\Affilfont{\itshape\footnotesize}
% \linespread{1.5}

% Define a custom note command for general notes
\newcommand{\mynote}[1]{\todo[color=yellow!40,inline]{#1}}

\DeclareMathOperator*{\argmin}{arg\,min}
% \DeclareMathOperator*{\argmax}{arg\,max}

% Nature Bibliography style
% \usepackage[backend=biber,style=nature]{biblatex}
% \addbibresource{library.bib} 
\bibliographystyle{unsrtnat}

% number equations by section
\numberwithin{equation}{section}

\title{Extreme Value Theory Notes}
\author{Patrick O'Toole}
% \date{July - September 2024}
\date{\today}

\begin{document}

\maketitle

\tableofcontents

\newpage

\todo{Limit number of warnings from chktex}
\todo{Replace notes in text with calls to mynote}
% \todo{Add macros for Y-i, a_mid_i, b_mid_i?}
% \todo{Get snippets working with vimtex(just use c-s)}
\todo{Add introduction, including stuff from Coles book?}
\todo{number equations}
\todo{Remove numbers for un-referenced equations}
\todo{Tidy bibtex references}
\todo{Include authors names rather than numbering for references}

\section{Introduction}\label{sec:intro}

\newpage
\section{Univariate extremes}\label{sec:uni}

\newpage
\section{Conditional extremes model}\label{sec:ce}

\todo{Look back into page 500 and equation 1.5, needed here?}
\todo{Uses beta for GPD scale, but beta used for normalising function, and want sigma for moments of Z, need to have consistent notation}
\todo{Improve equation referencing etc}

Description of model comes from \cite{Heffernan2004}, with the addition of using Laplace margins and constraining the normalising function parameters coming from \cite{Keef2013}.

\subsection{Data}

\begin{itemize} 
  \item Continuous vector variable $\bm{X} = (X_1, X_2, \ldots, X_d)$. 
  \item Want to estimate $\mathbb{P}(\bm{X} \in \bm{C})$, where $\bm{C}$ is an extreme set such that $\forall \bm{X} \in \bm{C}$, at least one component of $\bm{X}$ i extreme. 
  \item $C_i$ corresponds to the part of $C$ for which $X_i$ is the largest component of $\bm{X}$, by quantiles of the marginal distribution. 
  \item $C_i = C \cap \left\{ \bm{x} \in \mathbb{R}^d: F_{X_i}(x_i) > F_{X_j}(x_j); j = 1, \ldots, d; j \ne i \right\}$ for $i = 1, \ldots, d$., where $F_{X_i}$ is the marginal distribution function of $X_i$. % \todo{Double check capitalisation on Fs}
  \item Ignore subsets $C \cap \left\{ \bm{X} \in \mathbb{R}^d: F_{X_i}(X_i) = F_{X_j}(x_j) \text{for some} j \ne i \right\}$, as they are null sets. 
  \item C is an extreme set if all $x_i$-values in non-empty $C_i$ fall in upper tail of $F_{X_i}$, i.e.\ if $\nu_{X_i} = \inf_{\bm{x} \in C_i}{(x_i)}$, then $F_{X_i}(\nu_{X_i})$ is close 1 for $i = 1, \ldots, d$, so 
  \[
  \mathbb{P}(\bm{X} \in \bm{C}) = \sum_{i=1}^{d}{\mathbb{P}(\bm{X} \in C_i)} = \sum_{i=1}^{d}\textcolor{blue}{\mathbb{P}(\bm{X} \in C_i\mid X_i > \nu_{X_i})}\textcolor{red}{\mathbb{P}(X_i> \nu_{X_i})}
  \]
  \item \textcolor{red}{red probability is estimated with marginal extreme value model}, while the \textcolor{blue}{blue probability is estimated using an extreme value model for the dependence structure}. 
\end{itemize}

\subsection{Marginal extremes model}

\begin{itemize}
  \item Model marginal tail of $X_i$ with Generalised Pareto Distribution (GPD):
    \[
      \mathbb{P}(X_i > x + u_{X_i} \mid X_i > u_{X_i}) = {(1 + \xi_{i} x/\sigma_i)}_{+}^{-1/\xi_i}, x > 0 
    \]
    where $u_{X_i}$ is a high threshold, $\xi_i$ is the shape parameter, $\sigma_i$ is the scale parameter, and ${x}_{+} = \max(x, 0)$.
  \item Require a model for complete marginal distribution $F_{X_i}$  of $X_i$, so need to describe all $X_j$ values that can occur with any large $X_i$ value, which leads to the following piecewise semiparametric model:
    \[
      \hat{F}_{X_i}(x) = \begin{cases}
        1 - \{ 1 - \tilde{F}_{X_i}(u_{X_i})\} \left\{1 + \xi_i(x - u_{X_i})/\sigma_i\right\}_{+}^{-1/\xi_i} & \text{if } x > u_{X_i} \\
        \tilde{F}_{X_i}(x) & \text{if } x \le u_{X_i}
      \end{cases}
    \]
    where $\tilde{F}_{X_i}$ is the empirical distribution function of the $X_i$ values. 
  \item This gives us estimates of $\mathbb{P}(X_i < \nu_{X_i})$.
\end{itemize}

\subsection{Marginal transformation}

The conditional extremes model relies on a transformation of the marginal distributions to common margins in order to have an exponential upper tail.
These transformations are detailed below:

\subsubsection{Gumbel transformation}

\begin{align*}
  Y_i &= -\log[-\log\{\hat{F}_{X_i}(X_i)\}], i = 1, \ldots, d \\
      &= t_i(X_i; \phi_i, \tilde{F}_{X_i}(X_i)) \\
      &= t_i(X_i),
\end{align*}
where $\phi_i = (\sigma_i, \xi_i)$ are marginal parameters. \\
This gives $\mathbb{P}(Y_i \le y) = \exp(-\exp(-y)) \implies \mathbb{P}(Y_i > y) \sim \exp(-y) \text{ as } y \rightarrow \infty$, so $Y_i$ has an exponential upper tail. 

\subsubsection{Laplace transformation}
The Laplace transformation detailed in \cite{Keef2013} is given by
\[
  Y_i = \begin{cases}
    \log\{2F_{X_i}(x_i)\} &\text{ for } X_i < F_{X_i}^{-1}(0.5) \\
    -\log\{2[1 - F_{X_i}(x_i)]\} &\text{ for } X_i \ge F_{X_i}^{-1}(0.5) \\
  \end{cases}
\]
which means that 
\[
  \mathbb{P}(Y_i \le y) = \begin{cases}
    \exp(y)/2 &\text{ for } y < 0 \\
    1-\exp(-y)/2 &\text{ for } y \ge 0 \\
  \end{cases}
\]
so that both tails of $Y_i$ are exponential, and so for any $u > 0$, the distribution of $Y_i - u \mid Y_i > u$ and $(-Y_i + u) \mid Y_i \le -u$ are exponential with mean 1. 
This greatly simplifies the normalising functions seen in section \ref{subsubsec:norm}, as for Gumbel margins a more complex normalising function is required for negatively associated variables. 

\subsection{Asymptotic dependence}

\begin{itemize}
  \item 
    \[
      \lim_{y \rightarrow \infty}\{\mathbb{P}(\bm{Y}_{-i} \mid Y_i > y)\} = \begin{cases}
      0 &\text{for asymptotic independence} \\
      \ne 0 &\text{for asymptotic dependence}, 
    \end{cases}
    \]
  where $\bm{Y}_{-i} = (Y_1, Y_2, \ldots, Y_{i-1}, Y_{i+1}, \ldots, Y_d)$. 
  \item Existing methods for multivariate extremes (e.g.\ max-stable processes, copulas) can only model $\mathbb{P}(\bm{X} \in C)$ under asymptotic dependence.\todo{revisit this and talk a bit more about it!}
\end{itemize}

\subsubsection{Limit assumption}

\begin{itemize}
  \item For each $Y_i$, want to estimate $\mathbb{P}(\bm{Y}_{-i} \le \bm{y}_{-i} \mid Y_i = y_i)$ as $y \rightarrow \infty$. 
  \item We require the limiting distribution to be non-degenerate for all margins (see section\ref{sec:uni})
  \item Therefore, assume for every $i$ that there are vector normalising functions $\bm{a}_{\mid i}(y_i),\bm{b}_{\mid i}(y_i), \in \mathbb{R} \rightarrow \mathbb{R}^{(d-1)}$ such that for fixed $\bm{z}_{\mid i}$, 
    \[
      \lim_{y_i \rightarrow \infty}\{\mathbb{P}(\bm{Y}_{-i} \le \bm{a}_{\mid i}(y_i) + \bm{b}_{\mid i}(y_i)\bm{z}_{\mid i} \mid Y_i = y_i)\} = \bm{G}_{\mid i}(\bm{z}_{\mid i})
    \] \todo{revisit, probably wrong}
    where all margins of $\bm{G}_{\mid i}$ are non-degenerate, so 
    \[
    \lim_{z \rightarrow \infty}{\bm{G}_{j \mid i}(\bm{z})} = 1, \forall j \ne i
    \] (no mass at $+\infty$, some allowed at $-\infty$).
  \item Alternatively, the standardised variables
    \[
      \bm{Z}_{\mid i} = \frac{\bm{Y}_{-i} - \bm{a}_{\mid i}(y_i)} {\bm{b}_{\mid i}(y_i)}
    \]
    have the property that 
    \[
      \lim_{y_i \rightarrow \infty}\{\mathbb{P}(\bm{Z}_{\mid i} \le \bm{z}_{\mid i} \mid Y_i = y_i)\} = \bm{G}_{\mid i}(\bm{z}_{\mid i})
    \]
  \item Conditional on $Y_i > u_i$ as $u_i \rightarrow \infty$, $Y_i  - u_i$ and $\bm{Z}_{\mid i}$ are independent in the limit with limiting marginal distributions being exponential and $G_{\mid i}$ respectively:
    \[
      \mathbb{P}(\bm{Z}_{\mid i} \le \bm{z}_{\mid i}, Y_i - u_i = y_i \mid Y_i > u_i) \rightarrow G_{\mid i}(\bm{z}_i) \exp(-y) \text{ ,as } u_i \rightarrow \infty
    \]
  \item For each $j \ne i$, 
    \[
      Z_{j\mid i} = \frac{Y_j - a_{j\mid i}(y_i)}{b_{j\mid i}(y_i)} \sim G_{j\mid i}(z_{j\mid i}) \text { given } Y_i = y \text{ as } y_i \rightarrow \infty
    \]
    $\implies G_{j \mid i}$ is the marginal distribution of $G_{\mid i}$ associated with $Y_j$.
\end{itemize}

\subsubsection{Normalisation} \label{subsubsec:norm}

\begin{itemize}
  \item Under Gumbel margins, normalising functions have simple form for class of positively associated variable, but more complex for negatively associated variables:
    \todo{Fix, missing curly bracket}
    \begin{align*}
      \bm{a}_{\mid i}(y) &= \bm{\alpha}_i y + I_{\{\bm{\alpha}_{\mid i} = 0, \bm{\beta}_{\mid i} < 0\}}\{\bm{\gamma}_{\mid i} - \bm{\delta}_{\mid i}\log(y)\}, \\
      \bm{b}_{\mid i}(y) &= y^{\bm{\beta}_{\mid i}}, 
    \end{align*}
    where $\bm{\alpha}_{\mid i}, \bm{\beta}_{\mid i}, \bm{\gamma}_{\mid i}, \bm{\delta}_{\mid i}$ are vector constants, $\bm{\alpha}_{\mid i} \in [0, 1], \bm{\beta}_{\mid i} \in (-\infty, 1], \bm{\gamma}_{\mid i} \in (-\infty, \infty) , \bm{\delta}_{\mid i} \in [0, 1]$. 
  \item Under Laplace marginals, as in \cite{Keef2013}, the indicator term disappears, and we have that $\bm{\alpha}_{\mid i} \in [-1, 1]$. 
  \item $\alpha_{j \mid i}$ controls the level of association between $Y_j$ and large $Y_i$, with positive and negative values indicating positive and negative asymptotic dependence, respectively, and values closer to 0 indicating stronger asymptotic independence. 
  % \item $\beta_{j \mid i}$ controls the rate of convergence to the limiting
    % distribution, with values closer to 1 indicating slower convergences %
    % came from copilot, is this right??
  \item $\beta_{j \mid i}$ controls the spread $\ldots$ \todo{Fill in more formally}
\end{itemize}

\todo{Add interpretation of alpha and beta from Keef paper (page 400)}

\subsection{Conditional dependence model}

\begin{itemize}
  \item $\bm{G}_{\mid i}$ is modelled nonparameterically as the empirical distribution of 
    \[ 
      \bm{Z}_{\mid i} = \frac{\bm{Y}_{-i} - \bm{\alpha}_{\mid i}(Y_i)}{(Y_i)^{\bm{\beta}_{\mid i}}}
    \]
  \item All $d$ different conditional distributions are estimated separately.

  \item Our dependence model therefore is a multivariate semiparametric regression model of the form 
\begin{align} \label{eq:ce_model}
  \begin{split}
    \bm{Y}_{-i} &= \bm{a}_{\mid i}(\bm{y}_i) + \bm{b}_{\mid i}(\bm{y}_i)\bm{Z}_{\mid i} \\
                &= \bm{\alpha}_{\mid i}\bm{y}_i + \bm{y}_i^{\bm{\beta}_{\mid i}}\bm{Z}_{\mid i}, \text{ for } Y_i = y_i > u_{Y_i}.
  \end{split}
\end{align}

  with the use of Laplace margins and the definition of $a_{\mid i}$ and $b_{\mid i}$ from \cite{Keef2013}. 
  \mynote{This is the Conditional Extremes dependence model, the most important thing to know about it!}
\end{itemize}

\todo{on page 508, do I need to talk about Z hat? How G hat is its empirical distribution?}

\subsection{Extrapolation}

\todo{Change algorithm from Heffernan to Keef (simulates from Normal distribution)}
\begin{itemize}
  \item Can simulate from $\bm{X}|\bm{X}_i > \nu_{X_i}$ by simulating from $\bm{Y}|\bm{Y}_i > y_i$ and transforming back to $\bm{X}$ space, using the following algorithm:
    \begin{enumerate}
      \item Simulate $Y_i$ from the transformed marginal distribution conditional on exceeding $t_i(\nu_{X_i})$.
      \item Sample $\bm{Z}_{\mid i}$ from $\hat{G}_{\mid i}$, independently of $Y_i$. 
      \item Obtain $\bm{Y}_{-i} = \bm{\alpha}_{\mid i}(Y_i) + (Y_i)^{\bm{\beta}_{\mid i}}\bm{Z}_{\mid i}$.
      \item Transform $\bm{Y} = (\bm{Y}_{-i}, Y_i)$ back to the original scale by using the inverse of the marginal transformation. 
      \item the resulting vector $\bm{X}$ is a simulated value from $\bm{X} \mid X_i > \nu_{X_i}$.
    \end{enumerate}
  \item This algorithm can be used to estimate $\mathbb{P}(\bm{X} \in C_i \mid X_i > \nu_{X_i})$ by evaluating it as the long run proportion of the generated sample that falls in $C_i$. 
\end{itemize}

\subsection{Diagnostics}

\begin{itemize}
  \item Marginals are diagnosed as in univariate EVT (threshold with mean excess plot etc.\ , model fit assessed with probability and quantile plots)
  \item Dependence:
    \begin{itemize}
      \item Normalised variable $\bm{Z}_{\mid i}$ must have stable distribution over range of $Y_i$ used for estimation and evaluation. 
      \item Independence of $\bm{Z}_{\mid i}$ and $Y_i$ given $Y_i > u_i$ for high threshold $u_i$ is explored (similar to diagnostics for `ordinary' linear model)
      \item Can also use standard statistical tests for independence. 
    \end{itemize}
\end{itemize}

\subsection{Inference}

\begin{itemize}
  \item Inference for marginal parameters $\bm{\psi}$ and dependence parameters $\theta$ is done stepwise (loss of efficiency deemed small, estimation methods much easier)
\end{itemize}

\subsubsection{Marginal Estimation}

\begin{itemize}
  \item $d$ univariate distributions estimated jointly assuming independence between components in LL:
    \[
      \log\{L(\bm{\psi})\} = \sum_{i=1}^{d}{\sum_{j=1}^{n_{u_{X_i}}}{\log\{\hat{f}_{X_i}(x_{i \mid i, k})\}}}
    \]
    $f_{X_i}$ is the density associated with semiparametric marginal $\hat{F}_{X_i}$, and $n_{u_{X_i}}$ is the number of exceedances of $u_{X_i}$.
  \item Equivalent to fitting GPD for each margin to excesses over marginal threshold
  \item Can estimate GPD using more complicated methods, such as extreme value GAMs and spatiotemporal models which allow for covariates and information borrowing between marginals.
\end{itemize}

\subsubsection{Single conditional}

\begin{itemize}
  \item Want to estimate $\bm{\theta}_{\mid i}$ under minimal assumptions about $\bm{G}_{\mid i}$. \todo{Do I explicitely define theta and psi as parameters of G and marginals anywhere?}
  \item Assume $Z_{\mid i}$ has two finite marginal moments, $\bm{\mu}_{\mid i}$ and $\bm{\sigma}_{\mid i}$ (Note: $\bm{\sigma}_{\mid i}$ is the vector of standard deviations of $\bm{Z}_{\mid i}$, separate to parameter of GPD) \todo{Will have to change one of these, see Christian's paper (he uses psi and nu)}
  \item Therefore $\bm{Y}_{-i} \mid Y_i = y$ for $y > u_{Y_i}$ has vector mean and standard deviation
    \begin{align*}
      \bm{\mu}_{\mid i}(y) &= \bm{a}_{\mid i}(y) + \bm{\mu}_{\mid i}{\bm{b}_{\mid i}}(y), \\
      \bm{\sigma}_{\mid i}(y) &= \bm{\sigma}_{\mid i}{\bm{b}_{\mid i}}(y),
    \end{align*}
  \item (Note: $\bm{\mu}_{\mid i}(y)$ is the vector mean, while $\bm{\mu}_{\mid i}$ are it's respective parameters)
  \item $\implies$ $(\bm{\theta}_{\mid i}, \bm{\lambda}_{\mid i} = (\bm{\mu}_{\mid i}, \bm{\sigma}_{\mid i}))$ are parameters of multivariate regression model to be estimated. 
  \item Assume components of $\bm{Z}_{\mid i}$ are independent and Gaussian, for simplicity and convenience. 
  \item Independence assumption is reasonable as $\bm{\theta}_{\mid i}$ determines only the marginal behaviour of the conditional distribution. 
  \item The objective function for the point estimation of $(\bm{\theta}_{\mid i}, \bm{\lambda}_{\mid i})$ is 
    \[
      Q_{\mid i}(\bm{\theta}_{\mid i}, \bm{\lambda}_{\mid i}) = \sum_{j \ne i}\sum_{k=1}^{n_{u_{Y_i}}}{\left[\log\{\sigma_{j \mid i}(y_{i\mid i, k})\} + \frac{1}{2} \left\{ \frac{y_{j\mid i, k} - \mu_{j\mid i}(y_{j\mid i, k})}{\sigma_{j \mid i}(y_{i\mid i, k})} \right\}^2 \right]}
    \]
  \item Maximise jointly w.r.t $\bm{\theta}_{\mid i}$, $\bm{\lambda}_{\mid i}$, to obtain $\hat{\bm{\theta}}_{\mid i}$ with $\hat{\bm{\lambda}}_{\mid i} $ being nuisance parameters.
  \item For Gumbel margins, must estimate dependence model in two steps, first fixing $\gamma_{j \mid i} = \delta_{j \mid i} = 0$, then estimating both if the indicator function in equation hmm is satisfied.\todo{reference correct equation}
\end{itemize}

\subsubsection{All conditionals}

\begin{itemize}
  \item Falsely assume independence between different conditional distributions: 
    \[
      \hat{Q}( \bm{\theta}, \bm{\lambda}) = \sum_{i=1}^{d}{Q_{\mid i}(\bm{\theta}_{\mid i}, \bm{\lambda}_{\mid i})}
    \]
  \item False assumption above approximates the pseudo-likelihood as marginal density of $\hat{Y}_{-i}$ and conditional density of $\hat{Y}_{-i} \mid Y_i = y_i$ when $y_i < u_{Y_i}$ negligibly influence the shape of the pseudo-likelihood. 
  \item If variables all mutually asymptotically independent, then for sufficiently large $u_{Y_i}$ each datum will exceed at most one threshold, so independence assumption will be satisfied. 
\end{itemize}

\subsection{Uncertainty estimation (bootstrap)}

\begin{itemize}
  \item Uncertainty comes from semiparametric marginal models, normalising functions and distributions of residuals. 
  \item Semiparametric bootstrap used to evaluate standard errors of model parameter estimates and other estimated parameters such as $\mathbb{P}(\bm{X} \in \bm{C})$. 
  \item Assume marginal and dependence thresholds are fixed, therefore uncertainty due to threshold selection not accounted for. 
  \item Three stages: data generation under fitted model, estimation of model parameters, and derivation of estimate of parameters linked to extrapolation. 
  \item Two-step sampling algorithm used for data generation to replicate both the marginal and dependence features of the data. 
  \item \ldots
\end{itemize}

\todo{Finish this section}

\subsection{Return level}

When multivariate set $\bm{C}$ is described by single parameter $nu$ (i.e.\ $C = C(\nu)$), the return level $\nu_p$ for event with probability $p$ is defined as 
\[
  \mathbb{P}(\bm{Y} \in C(\nu_p)) = p
\]

\subsection{New constraints}

Some new constraints for the conditional extremes model were introduced in \cite{Keef2013}.

\begin{itemize}
  \item The coefficient of tail dependence between the pair of variables $(X_i, X_j)$ is given by
    \[
      \chi_{ij}(p, q) = \mathbb{P}\{X_j > F_j^{-1}(q) \mid X_i > F_i^{-1}(p)\} \text { for } p, q \in (0, 1)
    \]
  \item The limiting positive dependence between two variables is given by 
    \[
      \chi^+_{ij}\lim_{p \rightarrow 1}{\chi_{ij}(p, p)}
    \],
    while the limiting negative dependence is 
    \[
      \chi^-_{ij} = \lim_{p \rightarrow 1}{1-\chi_{ij}(1 - p, p)}
    \]
  \item Asymptotic positive dependence between pair $(X_i, X_j)$ gives $\chi^+_{ij} >0 1$, while asymptotic independence gives $\chi^+_{ij} = 0$.
  \item Similarly, asymptotic negative dependence gives $\chi^-_{ij} > 0$, while asymptotic independence gives $\chi^-_{ij} = 0$.
  \item We must preserve the bounds implied by the asymptotic dependence through a stochastic ordering of the conditional distributions of $Y_j \mid Y_i = y$ for large $y$ associated with asymptotic negative dependence, asymptotic independence, and asymptotic positive dependence, respectively. 
  \item Otherwise, the resulting joint probabilities can exceed the marginal probabilities, i.e.\
    \[
    \hat{\mathbb{P}}(X_i > F_i^{-1}(p),X_j > F_j^{-1}(q)) > \max(1-p, 1-q)
    \]
  \item Let the $q^{th}$ conditional quantile of $Y_j \mid Y_i = y$ for large $y$ be $y_{j \mid i}(q)$ and quantiles under the assumption of asymptotic positive and negative dependence be $y^+_{j \mid i}(q)$ and $y^-_{j \mid i}(q)$, respectively. 
    Then
    \begin{equation} \label{eq:ordering}
      y^-_{j \mid i}(q) \le y_{j \mid i}(q) \le y^+_{j \mid i}(q)
    \end{equation}
    for
    \begin{align*}
      y^-_{j \mid i}(q) &= -y + \bm{Z}_{j \mid i}^-(q), \\
      y_{j \mid i}(q) &= \alpha_{j \mid i}y + y^{\beta_{j \mid i}}(q), \\
      y^+_{j \mid i}(q) &= y + \bm{Z}_{j \mid i}^+(q),
    \end{align*}
    where
    $\hat{G}^-_{j\mid i}\{Z^-_{j \mid i}(q)\} = 
    \hat{G}_{j\mid i}\{Z_{j \mid i}(q)\} = 
  \hat{G}^+_{j\mid i}\{Z^+_{j \mid i}(q)\} = q$, where the $\hat{G}^-_{j \mid i}, \ldots$ are the estimated empirical distributions of $\bm{Z}_i$ for $Y_i > u$ under the assumption of asymptotic negative dependence, etc.
  \item Under the assumption of the asymptotic dependence for $(Y_i, Y_j)$ with $Y_i > u$, $Z^+_{j \mid i}(q)$ is the empirical $q^{th}$ quantile of $Z^+_{j \mid i} = Y_j - Y_i \text{ for } Y_i > u$. \\
  \item Similarly, $\ldots Z^-_{j \mid i}(q) = Y_j + Y_i$. \\
  \item Under the conditional extremes model, we estimate $Z_{j \mid i}(q)$ as the empirical $q^{th}$ quantile of $Z_{j \mid i} = (Y_j - \alpha_{j \mid i}Y_i) / Y_i^{\beta_{j \mid i}}$
  \item The theorem governing this ordering constraint is given by: \\
  \todo{Fix theorem environment}
  % \begin{theorem}
    For $\nu > u$, the ordering constraint \ref{eq:ordering} holds for all $y > \nu$ iff both Case 1 and II hold. \\
    \textbf{Case 1}: Either 
    \[
      \alpha_{j \mid i} \le \min\{
      1, 
      1 - \beta_{j \mid i}Z_{j \mid i}(q)\nu^{\beta_{j \mid i} - 1}, 
      1 - \beta_{j \mid i}Z_{j \mid i}(q) + \nu^{-1}Z^+_{j \mid i}(q)
      \}
    \]
    or 
    % \todo{Reference constraint above which orders ys}
    \todo{Finish theorem! Actually painful to type out lol}
  % \end{theorem}
  \item We only impose constraints on extrapolation to give greatest flexibility $\implies$ take $v$ to be above the maximum observed $Y_i$. 
  \item The constraints are built into the inference by having the profile likelihood for $(\alpha_{j \mid i}, \beta_{j \mid i})$ obtained by having the likelihood equal 0 if Theorem 1 is not satisfied.
  \item Also reduces variance of estimators by removing inconsistent estimates. 
\end{itemize}

\subsubsection{Application}

\todo{Do this section}

\subsection{Comparisons to other methods}
\todo{Do this section, try summarise, no need for too much detail can also use what other papers citing this one have said!}

In \cite{Tawn2018}, the conditional extremes model is compared with other commonly used multivariate extremes models, namely max-stable, Pareto and Gaussian processes $\ldots$

\subsection{Spatial conditional extremes}

\todo{Write this section!}
In \cite{Wadsworth2018}, the conditional extremes model is extended to the spatial domain. 

\todo{Also talk about spatio-temporal conditional extremes from Simpson INLA paper?}

\newpage
\section{Applications of conditional extremes model}\label{sec:ce_applications}

There have been many papers which have used the Conditional Extremes model in the context of multivariate extremes. 

\paragraph{Estimating the probability of widespread flood events}
In \cite{Keef2012_flooding}, $\ldots$

\paragraph{Joint modelling of extreme ocean environments incorporating covariate effects}

\paragraph{Modelling the effect of the El Niño-Southern Oscillation on extreme spatial temperature events over Australia}

\paragraph{High-dimensional modeling of spatial and spatio-temporal conditional extremes using INLA and Gaussian Markov random fields}

\paragraph{Model-based inference of conditional extreme value distributions with hydrological applications}

\newpage
\section{Bayesian spatial clustering for extremes} 
\todo{Make this a subsection of a clustering section!}

This algorithm for Bayesian spatial clustering of (hydrological) extremes is detailed in \cite{Rohrbeck2021}.
\begin{itemize}
  \item Clustering is done for two main reasons:
    \begin{itemize}
      \item interpretation
      \item improve inference by pooling information over similarly distributed variables.
    \end{itemize}
  \item The first is usually done for extremes by fitting multiple marginal GPDs and applying generic clustering techniques (k-means, k-mediods, etc) on some summary statistic (such as the scale parameter, $\sigma$). 
  \item The second often uses hierarchical modelling. Naturally in extreme analyses we have a lack of data, so pooling similar sites etc.\ for parameter estimation is highly advantageous. 
  \item Existing spatial clustering focuses on either the marginal distributions or the dependence structure. 
  \item This Bayesian clustering algorithm combines both ((1) GPD and (2) $\chi$) into likelihood, with reversible jump MCMC algorithm used to estimate cluster allocation and cluster specific marginal parameters. 
\end{itemize}

\subsection{Model}

\subsubsection{Data}

\begin{itemize}
  \item K sites with spatial locations $s_1, \ldots, s_k \in \mathbb{R}^2$.
  \item For areal data, $s_i$ is the centroid of the $k^{th}$ areal unit. 
  \item For geostatistical data, $s_i$ is the point location of $k^{th}$ site. 
  \item Distance $d_{k, k'} \ge 0$ between sites $k$ and $k'$.
  \item Declustering performed for each site, as data is seasonal and spatio-temporally dependent - only use highest observation per subperiod for which sitewise data is broken into..
  \item $R_{k, 1}, \ldots, R_{k, t}$ denotes the time series for site $k$ after declustering. 
  \item Assumed independent $\forall t \ne t'$. 
  \item $K$ latent variables $\bm{Z} = (Z_1, \ldots, Z_k)$, where $Z_k$ is the latent variable representing the cluster membership for site $k$.
  \item Let $J \in \{1, \ldots, K\}$ represent the number of clusters (so $J$ is also a random variable!).
  \item Cluster based on (1) similar marginal distributions and (2) spatial dependence (represented with $\chi_{k, k'}$) being greater between sites in the same cluster than sites in different clusters. 
\end{itemize}

\subsubsection{Marginal Model}

\begin{itemize}
  \item $R_{k, t} - u_k \mid R_{k, t} > u_k \sim \text{GPD}(\psi_k, \nu_k)$, so we have site-specific marginal parameters and thresholds. 
  \item In clustering, want cluster-specific, rather than site-specific GPD parameters, i.e.\ 
  \[
    R_{k, t} - u_k \mid (Z_k = j, R_{k, t} > u_k) \sim \text{GPD}(\sigma_j, \xi_j).
  \]
  Note here that we still have site-specific thresholds. \todo{Need to unify notation here}
\item Parameters of marginal model given $\bm{Z}$ denoted by $\bm{\theta}_m^{(J)} = \{\bm{\sigma}^{(J)}, \bm{\psi}^{(J)}\}$, where $\bm{\sigma}^{(J)} = (\sigma_1, \ldots, \sigma_J)$ and $\bm{\psi}^{(J)} = (\psi_1, \ldots, \psi_J)$. \todo{have J superscript within bm call}
\end{itemize}

\subsubsection{Dependence Model}

\mynote{Here, we will replace the use of $\chi$ with the semiparametric conditional extremes model for dependence in equation \ref{eq:ce_model}}

\begin{itemize}
  \item Rather than model full joint distribution over extreme events at sites, model $\chi_{k, k'}$ for pairwise extremal dependence ($\chi_{k, k'}$ defined above) \todo{Have section on multivariate extremes which has a definition for chi}
  \item $\chi_{k, k'} \in [0, 1]$ gives the limit probability of site $k$ observing an extreme event given site $k'$ recording one. 
  \item Constrain $\chi_{k, k'}$, conditional on $\bm{Z}$ such that expected value is larger within clusters than for sites in different clusters: 
  \begin{equation} \label{eq:chi_exp}
    \mathbb{E}(\chi_{k, k'} \mid Z_k = Z_{k'}) \ge \mathbb{E}(\chi_{k, k'} \mid Z_k \ne Z_{k'})
  \end{equation}
  \item Further constrain so that $\chi_{k, k'}$ decreases with increasing distance between sites $d_{k, k'}$, with exponential decay which is also faster for sites in different clusters:
  \begin{equation} \label{eq:chi_dist}
    \mathbb{E}(\chi_{k, k'} \mid Z_k = j, Z_{k'} = j') = \begin{cases}
      \exp(-\gamma_j d_{k, k'}) &\text{if } Z_k = Z_{k'} = j \\
      \exp(-\gamma_0 d_{k, k'}) &\text{if } Z_k \ne Z_{k'}, \\
    \end{cases}
  \end{equation}
  where $\gamma_0 > \max(\gamma_1, \ldots, \gamma_J) \ge 0$, which is ensured by introducing parameters $(\epsilon_1, \ldots, \epsilon_J), \epsilon_j \ge 0$ and constraining the cluster specific decay factors s.t.\ $\log(\gamma_j) = \log(\gamma_0) - \epsilon_j$.
\item $\chi_{k, k'} \mid \bm{Z} \in [0, 1]$ may differ between two pairs of sites and within the same cluster with same $d_{k, k'}$ (due to factors like topology), so we choose a beta distribution model with 
  \todo{Ask Christian about use of Beta distribution here}
  \begin{equation} \label{eq:beta_mod}
    \chi_{k, k'} \mid \bm{Z} \sim \begin{cases}
      \text{Beta}\left(\frac{\beta \exp(-\gamma_j d_{k, k'})}{1 - \exp(-\gamma_j d_{k, k'})}, \beta\right) \text{ if } Z_k = Z_{k'} = j \\
      \text{Beta}\left(\frac{\beta \exp(-\gamma_0 d_{k, k'})}{1 - \exp(-\gamma_0 d_{k, k'})}, \beta\right) \text{ if } Z_k \ne Z_{k'} \\
    \end{cases}
  \end{equation}
  which has expectation equal to \ref{eq:chi_exp}. 
  \item $\beta > 0$ is inversely proportional to the variance of $\chi_{k, k'}$.
  \item Therefore, dependence parameters are $\bm{\theta}_D^{(J)} = \gamma_0, \bm{\epsilon}^{(J)}, \beta\}$. \todo{Why not gamma j? Do we only need to estimate the epsilons that relate them to gamma 0?} 
  \item $\chi_{k, k'}$ only considered for adjacent sites.
  \item For geostatistical data, we derive the Voronoi partition of area and define sites as being adjacent if Voronoi cells are. \mynote{Could be expanded with SPDE approach from INLA? Interesting compromise though}
\end{itemize}
  
\subsection{Inference}

\begin{itemize}
  \item Bayesian inference used to estimate $J, \bm{Z}, \bm{\theta}_m^{(J)}, \bm{\theta}_D^{(J)}$, using declustered data $\bm{D} = \{r_{k, 1}, \ldots, r_{k, t}\}; k = 1, \ldots, K$.
  \item For marginal, data is marginal exceedances over threshold while for dependence the ranks of variables are used. 
  \item Because of this \textcolor{red}{difference in data}, inference for both sets of parameters is largely independent, and we can approximate the joint likelihood using the independent decomposition:
  \begin{equation} \label{eq:joint_likelihood}
    L(\bm{\theta}_m^{(J)}, \bm{\theta}_D^{(J)} \mid \bm{D}, \bm{Z}) = L_m(\bm{\theta}_m^{(J)} \mid \bm{D}, \bm{Z}) L_D(\bm{\theta}_D^{(J)} \mid \bm{D}, \bm{Z}).
  \end{equation}
\end{itemize}

\subsubsection{Marginal component}

\begin{itemize}
  \item Assuming thresholded data independent over all sites:
    \[
      L_m(\bm{\theta}_m^{(J)} \mid \bm{D}, \bm{Z}) = \prod_{k=1}^{K}{\prod_{\{t:r_{k, t} > u_k\}}{\frac{1}{\sigma_{Z_k}}\left( 1 + \xi_{Z_k} \frac{r_{k, t} - u_k}{\sigma_{Z_k}} \right)_+^{-1/\xi_{Z_k} -1}}}
    \]
    \todo{Ask Christian why the subscript is Zk here, rather than just k?}
  \item However, assuming spatial independence is not a valid assumption, as severe weather events usually affect a number of sites simultaneously. \mynote{There could be an option to include this additional constraint to the marginal component of the likelihood in an R package}
  \item General theory for asymptotic distribution of MLE $\hat{\bm{\theta}}_m^{(J)}$ for GPD parameters under independence assumption is 
    \[
      \sqrt{T}(\hat{\bm{\theta}}_m^{(J)} - \bm{\theta}_m^{(J)}) \sim \text{N}(0, \Sigma = H(\bm{\theta}_m^{(J)})^{-1}V(\bm{\theta}_m^{(J)})H(\bm{\theta}_m^{(J)})^{-1}),
    \]
\end{itemize}
  where $\bm{\theta}_m^{(J)}$ are the true parameter, H denotes the Fischer information (see paper for full definition).
  \item Using $L_m(\bm{\theta}_m^{(J)} \mid \bm{D}, \bm{Z})$ would underestimate the variance of $\bm{\theta}_m^{(J)}$ due to spatial dependence.
  \item Adjust likelihood around it's mode:
    \[
      L_m^{\text{adj}}(\bm{\theta}_m^{(J)} \mid \bm{D}, \bm{Z}) = L_m(\bm{\theta}_m^{(J)} + \bm{B}(\bm{\theta}_m^{(J)} - \hat{\bm{\theta}}_m^{(J)}) \mid \bm{D}, \bm{Z}),
    \]
    where the $2J \times 2J$ matrix $B$ depends on $\bm{Z}$ and is
    \todo{Fix}
    % \[
    %   \bm{B} = \left\{\[H(\bm{\theta}_m^{(J)})\]^{1/2}\right\}^{-1}
    % \]
  \item B is a block matrix of $J 2 \times 2$ blocks for each clustering, allowing for efficient computation. 
  \item Take $ L_m(\bm{\theta}_m^{(J)} \mid \bm{D}, \bm{Z}) =  L_m^{\text{adj}}(\bm{\theta}_m^{(J)} \mid \bm{D}, \bm{Z})$.
  \mynote{i.e.\ use the adjusted likelihood hereafter for the marginal component of the likelihood}
  
\subsubsection{Dependence component}
\begin{itemize*}
  \item Assume $\exists \tilde{u} \in [0, 1) \text{ s.t.\ } \chi_{k, k'}(u) = \chi_{k, k'} \forall \tilde{u} < u< 1$, i.e.\ it's equal to it's limit form, for all pairs of sites $(k, k')$.
  Let
  \item \[
      mathcal{Q}_{k, k'} = \{t: \hat{F}_k(r_{k', t} > \tilde{u}\}
  \]
  be the set of times when there is an exceedance of the \textcolor{red}{quantile threshold} at both sites $k$ and $k'$.
  \item The cardinality of this set is $Q_{k, k'} = |\mathcal{Q}_{k, k'}|$.
  \item Let $P_{k, k'} = #\{ t \in \mathcal{Q}_{k, k'} : \hat{F}_k(r_{k, t}) > \tilde{u}\}$ \todo{Ask Christian how P is different to Q?}
  \item $\implies$ $P_{k, k'} \sim \text{Binomial}(Q_{k, k'}, \chi_{k, k'})$.
  \item Estimate for $\chi_{k, k'}$ is $\hat{\chi}_{k, k'} = P_{k, k'} / Q_{k, k'}$, which is the proportion of exceedances of the $100\tilde{u}\%$ for site $k'$ that also exceed this quantile threshold for site $k$. 
  \item Combining this and Beta model for $\chi_{k, k'} \mid Z$ in equation \ref{eq:beta_mod}, and then integrating over $\chi_{k, k'}$, we obtain that
    \begin{equation} \label{eq:beta-binom}
    P_{k, k'} \mid \bm{D}, \bm{Z} \sim \begin{cases}
      \text{Beta-binomial}\left(Q_{k, k'}, \frac{\beta}{\exp{\gamma_j d_{k, k'}}}, \beta\right) &\text{if } Z_k = Z_{k'} = j \\
      \text{Beta-binomial}\left(Q_{k, k'}, \frac{\beta}{\exp{\gamma_0 d_{k, k'}}}, \beta\right) &\text{if } Z_k \ne Z_{k'} \\
    \end{cases}
  \end{equation}
  \item This model to hold only for pairs of adjacent sites.
  \item Denote the density function in \ref{eq:beta-binom} by $g$, then for a pair of adjacent sites $(k, k')$ the likelihood contribution to $L_D(\bm{\theta}_D^{(J)} \mid \bm{D}, \bm{Z})$ is
    \[
      L_D\left(\bm{\theta}_D^{\left(J\right)} \mid \bm{D}, \bm{Z}\right) = \left[g\left(P_{k, k'} \mid Q_{k, k'}, \bm{Z}, \bm{\theta}_D^{\left(J\right)}\right) \times g\left(P_{k', k} \mid Q_{k', k}, \bm{Z}, \bm{\theta}_D^{\left(J\right)}\right)\right]^{0.5},
    \]
    as we have two estimates for $\chi_{k, k'}$ which contain almost exactly the same information. 
  \item Under the assumption of independence for distinct pairs,
    \[
      L_D(\bm{\theta}_D^{(J)} \mid \bm{D}, \bm{Z}) = \prod_{k \sim k'}{L_D^{k, k'}}
    \]
  \item Likelihood is misspecified, $\ldots$ \todo{Complete}
\end{itemize*}

\subsection{Priors}\label{subsec:priors}
\begin{itemize}
  \item Priors required for $J, \bm{Z} \mid J, \bm{\theta}_m^{(J)}, \bm{\theta}_D^{(J)}$.
  \item $J \ge 1 \implies J -1 \sim \text{Poisson}(\kappa), \kappa \sim \text{Gamma}(1, 0.001)$ (weakly informative). 
  \item Clusters must be contiguous over space $\imples$ only give positive mass in prior to contiguous clusters. 
  \item $\bm{C}^{(J)} = \{C_1, \ldots, C_J\} \in \{1, \ldots, K\}, C_i \ne C_j \text{ if } i \ne j$ are centres of $J$ clusters, with each $k$ assigned to the closest cluster centre in terms of distance:
  \begin{equation} \label{eq:distance}
    Z_k \mid \bm{C}^{(J)} = \argmin_{j \in \{1, \ldots, J\}}{d_{k, C_j}}
  \end{equation}
  \item site $k$ assigned to cluster with lowest index if multiple centres minimise distance to the site (not well-defined otherwise). 
  \item Can assign prior to $\bm{Z} \mid J$ via $C^{(J)}$, which is given a uniform prior with 
    \[
      \mathbb{P}(C^{(J)} \mid J) = \frac{(K-J)!}{K!}
    \]
  \item With independent conjugate priors for the hyperparameters of both, we have that for $\bm{\theta}_m^{(J)}$:
    \begin{itemize}
      \item $\sigma_j \sim \text{LogNormal}(\mu^{\sigma}, \theta^{\sigma}), \mu^{\sigma} \sim \text{Normal}(0, 1), \theta^{\sigma} \sim \text{Inverse-Gamma}(1, 0.1)$.
      \item $\xi_j \sim N(\mu^{\xi}, \theta^{\xi}), \mu^{\xi} \sim \text{Normal}(0, 0.2), \theta^{\xi} \sim \text{Inverse-Gamma}(1, 0.1)$,
      \mynote{Could use penalised complexity prior for $\xi$, which would penalise the likelihood of having values differing greatly from 0 (implemented in INLA)}
    \end{itemize}
    and for $\bm{\theta}_D^{(J)}$:
    \begin{itemize}
      \item $\epsilon_j \sim \text{Exp}(\theta^{\epsilon}), \gamma_0 \sim \text{Exp}(0.001), \beta \sim \text{Exp}(0.001), \theta^{\epsilon} \sim \text{Gamma}(5, 2)$.
    \end{itemize}
\end{itemize}

\subsection{Reversible jump MCMC Algorithm}
\todo{Do this section!}

\begin{itemize}
  \item Want to sample from posterior distribution defined by the likelihood in \ref{eq:joint_likelihood} and the priors in \ref{subsec:priors}.
  \item The dimension of the parameter space changes with $J$, so use a reversible jump MCMC algorithm. 
  \item Given a current sample with $J$ clusters, propose one of the following seven moves:
    \begin{enumerate}
      \item \textbf{Birth}: Introduce a new cluster centre $C^*$ with parameters $\epsilon^*, sigma^* \text{ and } \xi^*$.
      \item \textbf{Death} Remove one of the existing cluster centres $C_1, \ldots, \C_J$. Sites previously allocated to the removed cluster are assigned to the other $J-1$ clusters according to the distance criterion in equation \ref{eq:distance}.
      \item \textbf{Shift} Move one cluster centre to nearby site which is not a cluster centre, update cluster labels $\bm{Z} \mid J$ according to equation \ref{eq:distance}. 
      \item \textbf{Sigma} Update $\bm{\sigma}^{(J)}$ for all clusters.
      \item \textbf{Xi} Update $\bm{\xi}^{(J)}$ for all clusters.
      \item \textbf{Chi} Update $\bm{\epsilon}^{(J)}, \gamma_0, \beta$.
      \item \textbf{Hyper} Update hyperparameters $\kappa, (\mu^{\sigma}, \theta^{\sigma}), \left(\mu^{\xi}, \theta^{\xi}, \theta^{\epsilon}\right)$.
    \end{enumerate}

  \item For a birth move, a new cluster centre $C^*$ is uniformly sampled from one of the $K - J$ sites which are not currently cluster centres. \mynote{Cluster centres are therefore ``mediods'' in some sense, in that they are actually observed sites, rather than some aggregation between sites}
  \item Index at which to insert $C^*$ is also uniformly sampled.
  \todo{Finish!}
\end{itemize}

\newpage
\section{Other methods for extremal clustering}

Similar to section \ref{sec:ce_applications}, this section will detail other methods for clustering of extremes. 

% Clustering falls into two categories, clustering to aid qualitative inference and clustering for improved parameter estimation. \\
% The first category usually uses some distance or divergence metric/measure, such as $\chi$, KL divergence or F-madogram to cluster sites based on a distance matrix. \\
% The second category usually uses hierarchical modelling to pool information over similarly distributed variables. \todo{Or otherwise?}

\paragraph{Clustering of maxima: Spatial dependencies among heavy rainfall in France}

In \cite{Bernard2013}, a clustering algorithm for (weekly precipitation) maxima is proposed which uses (an empirical/nonparametric estimator for) the F-madogram, a type of variogram (estimates spatial correlation of spatial random field) for extremes, and PAM (partitioning around mediods). \\
Summary:
\begin{itemize}
  \item bivariate vector $(M_i, M_j)^T$ assumed to follow a bivariate EVT distribution.
  \item A variogram of order $p$ is defined as the moment of order $p$ of the difference between maxima $M_i$ and $M_j$, $\matbb{E}|M_i - M_j|^p$.
  \item The F-madogram $d_{ij}$ is, amongst other things:
    \begin{enumerate}
      \item an interpretable distance, thus forming a distance matrix over which to cluster,
      \item expressed in terms of the scalar "extremal coefficient" $V_{ij}(1, 1)$, which gives information about the degree of dependence between $M_i$ and $M_j$. 
      \item forms a copula, and so is completely decoupled from marginal estimates, meaning there is no need to re-estimate a GEV distribution at each site, and it is not required to assume these maxima come from a GEV, but only that they lie in the domain of attraction of max-stable distribution. 
    \end{enumerate}
  \item PAM preferred to k-means, for the following reasons:
    \begin{enumerate}
      \item K-means averages over all cluster members. 
      The average of normally distributed obs remain Gaussian, but this is not the case for maxima following an extremal distribution.
      \item Taking mediods ensures that the cluster centre is an actual observation, which allows the maxima to remain maxima and does not apply any averaging or smoothing. 
    \end{enumerate}
  \item Interestingly, PAM not given geographical information, but still produces spatially coherent clusters. 
  \item Choice of K (number of clusters) and assessment of clustering quality made through analysis of silhouette coefficient, which compares cluster tightness (small distance within cluster) with cluster dissociation/separation (clusters should be adequately distinct).
\end{itemize}

\paragraph{Partitioning into hazard subregions for regional peaks-over-threshold modeling of heavy precipitation}

\todo{Read paper!}

\cite{Carreau2017} derives a model which uses covariates to estimate $\sigma(\bm{x})$ and conditional mixture of GPDs with subregions defined by constant shape parameters $\xi_j$, estimated through EM algorithm and partitioned into subregions, the number of which was determined via out-of-sample cross validation. \\
Summary:
\begin{itemize}
  \item As $T \uparrow$, $\xi$ becomes determining factor in return/hazard level estimates, particularly in how it determines the tail behaviour of the distribution.
  \item $\xi$ estimation is difficult, as it is highly variable and can be influenced by many factors, such as the choice of threshold, the number of observations, etc.
  \item Approach here is to treat hazard level as piecewise constant, and partition region of interest into subregions over which to fit a conditional mixture of GPDs (can be seen as extension of \cite{Cooley2007} which uses two subregions).
  \item \# regions sees bias-variance tradeoff; more regions means more $\xi$ to estimate with less data (more variance), while less regions means greater bias as each subregion will represent the local characteristics of more locations (more groups often also harder to interpret). 
  \item $\sigma$ doesn't have same issue, so just estimated with $\sigma(\bm{x})$, i.e.\ a function of covariates. 
  \item EM algorithm: \\
  \begin{itemize}
    \item \textbf{E-step} estimates partition $C$, uses probability weighted moment estimators, U-statistics, kernel regression and k-means (complex, see paper)
    \item \textbf{M-step} estimates $\mathbb{P}(C = j \mid \bm{x}) \text{ probability of subregion membership used in mixture GPD}, \sigma(\bm{x}), \xi_j$ for each subregion $j$.
  \end{itemize}
  \item Number of subregions chosen by out-of-sample CV using three different loss fuctions, including Anderson-Darling statistic (more sensitive to changes in the tails)
\end{itemize}


\paragraph{Clustering bivariate dependencies of compound precipitation and wind extremes over Great Britain and Ireland}

\cite{Vignotto2021} k-mediods algorithm of KL divergence between events characterised by risk function (sum/max) for Pareto transformed extreme wind and precipitation observations over Great Britian and Ireland. \\ 
Summary:
\begin{itemize}
  \item \textbf{Data}: Weekly sum of precipitation and average of daily wind speed maxima used, as precipitation and wind speed extremes can be linked through storms with a lag of several days due to persistent weather patterns. 
  \item \textbf{Mapping from bivariate to univariate space}: \todo{Word better}
  \begin{itemize}
    \item Marginal distributions (i.e.\ rain and wind speed at a single site) are transformed to standard Pareto distributions. 
    \item Risk function computed on Pareto scale $r:\mathbb{R}^2 \rightarrow \mathbb{R}$ used to define which points are extreme, where $r(x, y) = x + y$ or $r(x, y) = \max(x, y)$, mapping from bivariate to univariate space (taking (Pareto transformed) wind and precipitation estimates at each site and giving us a single estimate for each)   \end{itemize}
  \todo{is the Risk function only used to determine which points are extreme, with the original data bivariate data used to calculate KL divergence???}
  \textbf{Distance metric}:
  \begin{itemize}
    \item coefficient of tail dependence $\chi/\bar{\chi}$ cannot directly measure similarity of extremal behaviour of two bivariate random variables $\bm{X}^{(1)} = (X_1^{(1)}, X_2^{(1)})$ and $\bm{X}^{(2)} = (X_1^{(2)}, X_2^{(2)})$, i.e.\ rain and wind speeds at two different locations.
    \item Instead, use the Kullback-Leibler divergence, which measures the difference between two probability distributions $P$ and $Q$, and is defined as 
    \[
      D_{KL}(P \mid Q) = \int{p(x) \log\left(\frac{p(x)}{q(x)}\right)dx}.
    \]
    KL divergence is shown to generalise the concept of $\chi$, since $\ldots$ \todo{Finish}
  \item Extreme points $\{R^{(j)} > q_u^{(j)}\}$ partitioned into $W = 3$ sets, one for co-occurring extremes and two for where data is extreme for only one variable (easily extended to multivariate case from bivariate).
  \item Empirical proportions of data points belonging to each set used to estimate KL divergence between any two sites (i.e.\ how similar occurrence of extremes are at two sites), this giving distance/dissimilarity matrix over which to cluster.
  \end{itemize}
  \item K-mediods algorithm clusters sites over KL divergences. 
  \item Silhouette coefficient used to choose number of cluster and assess solution. 
  \item Makes interesting conclusions about bivariate extremal behaviour of Ireland which will be useful for TFR report, but is different in it's use of gridded data, which underestimates extremal precipitation and wind speeds. 
\end{itemize}



\paragraph{Modelling panels of extremes}

\cite{Dupuis2023} derives EM (really MM) algorithm for identifying group structure and group-specific model parameters for GEV distributed panel data. \\
Summary:
\begin{itemize}
  \item \textbf{Data}: data for individuals/locations $i$, time $t$ (known as panel data, i.e.\ $X_{i t}$) which are GEV distributed. 
  \item Because of the nature of extremes, both complete and no pooling results in poor parameter estimates, so some/"partial" pooling over similar/homogeneous locations desired.
  \item Use EM/EE algorithm and QML (where variance-covariance matrix may be misspecified) to iteratively estimate (i.e.\ estimation is disentangled for simplicity) group structure/assignment $\bm{\tau}$ and group-specific parameters $\bm{\theta}$ related to parameters of GEV through individual regression equations (so each GEV parameter estimated separately). 
  The consistency of this algorithm is also proven.
  \item grouping is latent and ``data-driven'', rather than based on domain knowledge, and is done mainly for improved parameter estimation, as opposed to methods which derive distance matrices over which to perform e.g.\ k-mediods clustering. 
  \item Stronger dependence in data helps group identification because it reduces the variance among individuals in the same group, but it gives worse quantile estimates (i.e.\ stronger dependence gives lower effective sample size $\implies$ less information $\implies$ greater variance in QML estimator).
  \item Simulation study performed, and applications to financial risk, extreme temperature and flood risk data shown to be effective (and better than when using groupings based on domain knowledge). 
\end{itemize}

\paragraph{Similarity-based clustering for patterns of extreme values}

\cite{deCarvalho2023} uses k-means to cluster over bivariate cluster centroid of the extremal index and the heteroskedastic function, interpreted as the magnitude and frequency of extreme events, respectively. 
Summary (from reading course report):
\begin{itemize}
  \item Heteroskedastic extremes violate assumption of IID observations, and may exhibit serial dependence or be drawn from different distributions. 
  \item The \textbf{heteroskedastic function} $c$ gives the frequency of extremes, with $c = 1$ defining "homoskedastic extremes". 
  Defined as a limit which compares only the distribution tails, not imposing any assumption on centre of distributions. 
  \item \textbf{Extremal index} $\gamma$ (\textbf{the same as the shape parameter!}) is a scalar which controls the behaviour of a CDF in its right tail (i.e.\ it's rate of tail decay), and is often thought of as the inverse of the limiting mean cluster size).
  \item Nonparametric kernel-based estimator and Hill function provide respective estimates $\hat{c}, \hat{\gamma}$.
  \item Two quantities are jointly thresholded such that a specific quantile of observations is preserved. 
  \item The level of bias towards one of these metrics is parameterised in this procedure, so that one can be favoured over the other in an analysis, as deemed necessary. 
  \item Standard k-means clustering performed on bivariate cluster centroid $(\hat{c}, \hat{\gamma})$.
\end{itemize}

\newpage
\section{Bayesian extremes}

Bayesian statistics has seen widespread adoption in the context of Extreme Value Theory. 
Naturally, as parameter and estimation uncertainty is high, the use of priors to incorporate expert knowledge, as well as hierarchical modelling to share information, is highly advantageous.
Here, we will detail some of the methods used in Bayesian extremes.

\paragraph{Bayesian analysis of extreme values by mixture modeling}

% \cite{Bottolo2003} derives a mixture model of Poisson processes $PP(\mu, \sigma, \xi)$ with priors on PP parameters which have parameter-specific group structuring, allowing for high flexibility
\cite{Bottolo2003} defines exceedances over a given threshold as generated by a model characterised by a Poisson process $PP(\mu, \sigma, \xi)$, and derives a hierarchical mixture prior for each of these parameters which has (unknown, latent) parameter-specific group structuring (allowing for great flexibility) estimated through a Reversible Jump MCMC (RJMCMC) scheme similar to that of \cite{Rohrbeck2021}. 
\begin{itemize}
  \item Data is exceedances over a threshold, which are modelled as a (Poisson) point process with parameters $\mu, \sigma, \xi$.
  \item incorporating prior knowledge in context of extremes useful due to rarity of extremal data.
  \item Proposed Bayesian hierarchical model with (parameter-specific) grouping of type-effects, where grouping is latent and ``data-driven'', as in \cite{Dupuis2023}. 
  \item RJMCMC algorithm used to estimate group structure and group-specific parameters.
  \item Previous models assumed exchangeability of parameters meaning that types were treated symmetrically and correspondingly parameter estimates were shrunk towards common points. 
  \item Under mixture priors, parameters are assumed to be i.i.d.\ according to some finite-mixture distribution with number of components $w$, corresponding weights $k$ hyperparameters $\delta$, and latent variable $Z_i$ indicating the mixture component to which the parameter belongs.
  \item Hyperpriors perform partial-pooling between groups, with $k \sim \mathbb{P}(k^{\eta} = k)$ (can be uniform, fully specified or somewhere in between) for a PP parameter $\eta$ and Dirichlet prior on $w^{\eta} \mid k^{\eta}$ (nice DAG for these in paper). 
\end{itemize}

\paragraph{Bayesian spatial modeling of extreme precipitation return levels}

\begin{itemize}
  \item Abstract/summary:
    \begin{itemize}
      \item \cite{Cooley2007} models \textbf{r-year return levels} (with uncertainty) for extreme precipitation in Colorado.
      \item Separately hierarchical models for intensity $\mathbb{P}(Z(\bm{x}) > z + u \mid Z(\bm{x}) > u)$ and frequency $P(Z(\bm{x}) > u$) at location $\bm{x}$ under GPD and binomial distributions, respectively, as in chapter 5 of \cite{Coles2001}. 
      \item Both models incorporate latent spatial process characterised by geographical and climatological covariates using a Gaussian process. 
      \item MCMC and spatial interpolation used for inference. 
    \end{itemize}
    \item temporal dependence reduced by declustering (keeping only highest of consecutive days exceeding threshold). 
    \item Work in a ``climate space'' (rather than longitude/latitude), where the coordinates of $\bm{x}$ are defined by orographic and climatological measures. 
    \item Both models are described by the three layer hierarchical model
      \[
        p(\bm{\theta} | \bm{Z}(\bm{x})) \propto p_1(\bm{Z}(\bm{x}) | \bm{\theta}_1) p_2(\bm{\theta}_1 | \bm{\theta}_2) p_3(\bm{\theta}_2),
      \]
      where the first layer models the data (with a GPD or Binomial), the second layer models the latent process (with a Gaussian process) and the third layer consists of the hyperparameters for the parameters $\theta_2$ that drive the latent process (with $\theta_1$ being the parameters of the GPD/binomial likelihoods). 
    \item Best model amongst different covariate forms for $\phi = log(\sigma) and \xi$ were compared via DIC (also using long/lat vs climate space), $\xi$ taken to be different for ``mountainous'' and ``planes'' regions, so fit for just \textbf{two separate regions} (with shared information) rather than estimated for each site (important as this is quite like clustering/grouping for parameter estimation improvement, although the clustering is via domain knowledge rather than data-driven). 
\end{itemize}

\paragraph{A hierarchical max-stable spatial model for extreme precipitation}

\paragraph{Leveraging Extremal Dependence to Better Characterize the 2021 Pacific Northwest Heatwave}
\cite{Zhang2024}


% \printbibliography
\bibliography{library}

\end{document}
