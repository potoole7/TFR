\documentclass[10pt]{beamer}
% \usepackage[colorlinks=true,citecolor=blue, linkcolor=blue]{hyperref}
\usepackage{hyperref}
\usepackage[utf8x]{inputenc}
\usepackage{algorithm, algorithmic}
\usepackage[ruled,vlined]{algorithm2e}
% \usepackage{fontawesome}
\usepackage{graphicx}
% \usepackage[english]{babel}
\usepackage[nodayofweek]{datetime}

\usepackage[colorinlistoftodos]{todonotes}
\usepackage{graphicx}
\usepackage{mathtools}
\usepackage{amssymb}
\usepackage{amsmath}
\usepackage{bbm}
\usepackage{pdflscape}
\usepackage{caption}
\usepackage{subcaption}
\usepackage[T1]{fontenc}
% \usepackage[utf8]{inputenc}
\usepackage{authblk}
\usepackage{pdfpages}
\usepackage{setspace} 
\usepackage{booktabs}
\usepackage{longtable}
\usepackage{float}
\usepackage{tikz}
% \usepackage{multirow}
% \setlength{\tabcolsep}{5pt}
% \usepackage[parfill]{parskip}
% \renewcommand{\arraystretch}{1.5}

% ------------------------------------------------------------------------------
% Use the beautiful metropolis beamer template
% ------------------------------------------------------------------------------
\usepackage[T1]{fontenc}
\usepackage{fontawesome}
\usepackage{FiraSans} 
\mode<presentation>
{
  \usetheme[progressbar=foot,numbering=fraction,background=light]{metropolis} 
  \usecolortheme{default} % or try albatross, beaver, crane, ...
  \usefonttheme{default}  % or try serif, structurebold, ...
  \setbeamertemplate{navigation symbols}{}
  % \setbeamertemplate{caption}[numbered]
  %\setbeamertemplate{frame footer}{My custom footer}
} 

% ------------------------------------------------------------------------------
% beamer doesn't have texttt defined, but I usually want it anyway
% ------------------------------------------------------------------------------
\let\textttorig\texttt
\renewcommand<>{\texttt}[1]{%
  \only#2{\textttorig{#1}}%
}

% ------------------------------------------------------------------------------
% minted
% ------------------------------------------------------------------------------
\usepackage{minted}


% ------------------------------------------------------------------------------
% tcolorbox / tcblisting
% ------------------------------------------------------------------------------
\usepackage{xcolor}
\definecolor{codecolor}{HTML}{FFC300}

\usepackage{tcolorbox}
\tcbuselibrary{most,listingsutf8,minted}

\tcbset{tcbox width=auto,left=1mm,top=1mm,bottom=1mm,
right=1mm,boxsep=1mm,middle=1pt}

\newtcblisting{myr}[1]{colback=codecolor!5,colframe=codecolor!80!black,listing only, 
minted options={numbers=left, style=tcblatex,fontsize=\tiny,breaklines,autogobble,linenos,numbersep=3mm},
left=5mm,enhanced,
title=#1, fonttitle=\bfseries,
listing engine=minted,minted language=r}


% ------------------------------------------------------------------------------
% Listings
% ------------------------------------------------------------------------------
\definecolor{mygreen}{HTML}{37980D}
\definecolor{myblue}{HTML}{0D089F}
\definecolor{myred}{HTML}{98290D}

\usepackage{listings}

% the following is optional to configure custom highlighting
\lstdefinelanguage{XML}
{
  morestring=[b]",
  morecomment=[s]{<!--}{-->},
  morestring=[s]{>}{<},
  morekeywords={ref,xmlns,version,type,canonicalRef,metr,real,target}% list your attributes here
}

\lstdefinestyle{myxml}{
language=XML,
showspaces=false,
showtabs=false,
basicstyle=\ttfamily,
columns=fullflexible,
breaklines=true,
showstringspaces=false,
breakatwhitespace=true,
escapeinside={(*@}{@*)},
basicstyle=\color{mygreen}\ttfamily,%\footnotesize,
stringstyle=\color{myred},
commentstyle=\color{myblue}\upshape,
keywordstyle=\color{myblue}\bfseries,
}

% ------------------------------------------------------------------------------
% The Document
% ------------------------------------------------------------------------------

\begin{document}

\title{Developing clustering algorithms for conditional extremes models}

\date{
  \footnotesize Thesis formulation report presentation \\ 
  \today \\
  % \shortdate \\
  \vspace{0.4cm}
  \emph{Paddy O'Toole}
}

% \title{Wessex Water: Bayesian Network for Source Apportionment}
% \author{\emph{Sam Williams, Paddy O'Toole, Christian Rohrbeck, Haiyan Zheng, Emiko Dupont}}
% \date{\today}


\frame{\titlepage}

%==================================================

% \frame{\begin{small}
% \frametitle{Overview}
% \tableofcontents\end{small}}

%=================================================
\begin{frame}
\frametitle{Table of Contents}
\tableofcontents
\end{frame}

\section{Introduction}

% TODO: Perhaps add one more introduction slide?

% Motivation behind EVT
\begin{frame}{Introduction}
% \todo{Include use cases}
    % \begin{itemize}
    %     \item Extreme Value Theory is a field of statistics concerned with modelling the extremal tail of distributions, where standard methods do not perform well
    %     \item Many use cases, particularly popular for environmental data (rainfall, flooding, storms, temperature), where tail events often catastrophic. 
    %     \item Concerned with performing \textbf{extremal clustering}, helps with dimensionality reduction and understanding patterns in data
    %     % Need to have metric for magnitude over which to deem events as extreme, and then frequency or concommitant extremes for clustering 
    % \end{itemize}
\end{frame}

% MV extremes, uses, methods, shortcomings - clustering
\begin{frame}{Multivariate extremes}

\end{frame}

\section{Motivating example}

% Describe motivating example
\begin{frame}{Motivating example - Ireland}

\end{frame}

\section{Univariate extremes}


% GPD formula, threshold selection, non-stationary version
\begin{frame}{Generalised Pareto distribution}
\end{frame}

% Plot of sigma, xi values across Ireland
\begin{frame}{Motivating example}

\end{frame}

\section{Conditional extremes}
\begin{frame}{Introduction}

\end{frame}

% Univariate piecewise function
\begin{frame}{Univariate}

\end{frame}

% Transform to Laplace
\begin{frame}{Marginal transformation}
\end{frame}

% Multivariate piecewise function, Z, Y independent
\begin{frame}{Multivariate}

\end{frame}

% Methods for estimation, assumption on Z
\begin{frame}{Estimation}

\end{frame}

% Poor results initially, high uncertainty
\begin{frame}{Motivating example - Sensitivity}

\end{frame}

% Improved here
\begin{frame}{Fixing $\beta$}

\end{frame}

% Map of alpha values
\begin{frame}{Results}

\end{frame}

% Rain vs wind speed alpha vals, and vs lon/lat
\begin{frame}{Interpretation}

\end{frame}

% Briefly talk about extensions to model
\begin{frame}{Extensions to CE model}

\end{frame}

% TODO: May need more information/slides here
\section{Clustering}

% Reasons for and types of clustering 
\begin{frame}{Clustering}
\end{frame}

\begin{frame}{Explanatory clustering}
\end{frame}

\begin{frame}{Hierarchical clustering}
\end{frame}

\section{Conclusions and future work}

\begin{frame}{Conclusions}
\end{frame}

\begin{frame}{Future work}
\end{frame}

% TODO: Add references!

%============================================================
\frame{
\vskip30mm
\centerline{\Large\color{violet}\textsc{Thank you!}}
\centerline{\Large\color{violet}\textsc{Any Questions?}}
\vskip40mm
}





\end{document}
